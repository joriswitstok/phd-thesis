%!TEX root = ../thesis.tex
% ************************** Thesis Abstract *****************************
% Use `abstract' as an option in the document class to print only the titlepage and the abstract.
\begin{abstract}
    
    \noindent Current observational facilities, such as the Very Large Telescope (VLT), \textit{Hubble Space Telescope} (\textit{HST}), and Atacama Large Millimeter/submillimeter Array (ALMA), have enabled us to perform detailed spectroscopic analyses of distant galaxies well into the Epoch of Reionisation (EoR). This crucial phase transition witnessed baryonic matter, mostly in the form of cold, neutral hydrogen gas, being chemically enriched, ionised, and heated as a result of the formation of the first stars and galaxies. Here, I present the results of several studies aiming to shed light on the early evolutionary stages of galaxies and their contribution to Cosmic Reionisation. Using cosmological hydrodynamical simulations, I consider the prospects of mapping the intergalactic medium (IGM) in the most prominent hydrogen emission line, \lymana. Turning to observations, I present and analyse multiple spectroscopic datasets of individual high-redshift galaxies with the aim of understanding the process of star formation on the scale of the interstellar medium. Firstly, I show the spectroscopic measurements of a unique, strongly gravitationally lensed galaxy at redshift 5, taken by VLT/X-shooter and VLT/SINFONI, are consistent with a young, metal-poor, star-forming system with a hard radiation field. This galaxy is likely analogous to typical EoR galaxies, revealing \lymana\ and \MgII\ emission lines that may indicate the leakage of ionising photons into the IGM. Secondly, focussing on far-infrared and rest-frame UV observations of five UV-bright, star-forming galaxies at redshift 7 obtained with ALMA and \textit{HST} respectively, I show these measurements point towards similar physical properties, though there are hints of substantial metal enrichment in these systems. Constraints on the dust continuum of one source indicate the presence of a surprisingly cold and massive dust reservoir. Finally, I discuss directions for future work, in particular the synergy of existing observatories in combination with \textit{JWST}, the much-anticipated near- and mid-infrared space-based observatory that has recently started acquiring spectroscopy of the most distant galaxies.
    
    % Key words are inserted automatically from thesis.tex (turned on in class file)
    
\end{abstract}
