%!TEX root = ../thesis.tex
% ************************** Thesis Acknowledgements **************************

\begin{acknowledgements}      
    
    First of all, I would like to acknowledge the Kavli Institute for Cosmology, the Cavendish Laboratory, and the Institute of Astronomy for having made possible this research. Also, my gratitude goes out to the the developers of all software packages used for this research:
    \begin{enumerate}
        \item \program{python}, and several packages: the \program{SciPy} library \citep{Jones2001}, its modules \program{NumPy} \citep{2011CSE....13b..22V} and \program{Matplotlib} \citep{Hunter2007}, the \program{Astropy} \citep{2013A&A...558A..33A, 2018AJ....156..123A}, \program{pymultinest} \citep{2009MNRAS.398.1601F, 2014A&A...564A.125B}, and \program{DrizzlePac} packages (\url{https://www.stsci.edu/scientific-community/software/drizzlepac.html}.
        \item the CHIANTI atomic database. CHIANTI is a collaborative project involving George Mason University, the University of Michigan (USA), University of Cambridge (UK) and NASA Goddard Space Flight Center (USA).
        \item \program{SAOImage DS9} and \program{TOPCAT} (Tool for OPerations on Catalogues And Tables)
        \item The LaTeX distribution (CITE) %\LaTeX\
    \end{enumerate}
    
    Special thanks goes to Ewald, Girish, and Martin for all their assistance, useful advice, and enlightening discussions.
    
    I gratefully acknowledge financial support from the Fondation MERAC (Mobilising European Research in Astrophysics and Cosmology), European Research Council (ERC), and the Science and Technology Facilities Council (STFC) that have made this research possible. Furthermore, I am indebted to the support of all staff at the European Southern Observatory (ESO), National Radio Astronomy Observatory (NRAO), National Astronomical Observatory of Japan (NAOJ), Space Telescope Science Institute (STScI), European Space Agency (ESA), and National Aeronautics and Space Administration (NASA).
    
\end{acknowledgements}
