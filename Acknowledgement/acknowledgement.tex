%!TEX root = ../thesis.tex
% ************************** Thesis Acknowledgements **************************

\begin{acknowledgements}      
    
    I owe a great debt of thanks to the many people that have helped this thesis come to be. First and foremost, I should thank Renske and Roberto, a dynamic dream duo of supervisors. They have guided me not only through the many rewarding experiences of my PhD journey, but also past the inevitable obstacles I encountered along the way, notably including a global pandemic and a laptop meltdown in my final year. From a scientific point of view, it has been a true privilege to have been mentored by two astronomers so exceptionally knowledgeable, inventive, and insightful. Personally, I am immensely glad to have been able to work with such supportive, generous, and kind-hearted people.
    
    These qualities seem to extend to all people who have spent time at the Institute of Astronomy (IoA), or at least those that I have come across, starting (before I had even heard of the IoA) with Jarle Brinchmann. I am grateful for his supervision during my time in Leiden, which was an important inspiration for me to continue pursuing astrophysics. When I did arrive at the IoA for my master's degree, I managed to acquire as many as three wonderful supervisors: my sincere thanks go out to Ewald, Girish, and Martin for their patience during my first steps to research and for their continued guidance and advice leading up to, and during, my doctoral studies. I have been fortunate to meet many more incredible people at the IoA that have become close friends: Clay, Pooneh, and Sam are among those I know since the very first day of the program. Moving into the Kavli, I was lucky to share the illustrious K35 office with Asia, Connor, Lester, Yucheng, Federica, Giulia, and others. Discussions with the talented PhD students, postdocs, staff, and many visitors at the Kavli and IoA have been hugely inspiring, something I became acutely aware of due to a distinct lack of discussion groups and coffee breaks during the pandemic closures. A special mention goes to Steve for his unwavering and indispensable presence at the Kavli. Further, I am particularly thankful for the insightful and stimulating feedback from Lewis, Mirko, Gareth, Nimisha, Nicolas, and recently Francesco, Tobias, and Sandro.
    
    For these reasons, I am grateful to the Kavli Institute for Cosmology, the Cavendish Laboratory, the IoA, and the wider University for accommodating these invaluable interactions that have ultimately shaped the research presented in this dissertation. I gratefully acknowledge financial support from the Fondation MERAC (Mobilising European Research in Astrophysics and Cosmology), European Research Council (ERC), and the Science and Technology Facilities Council (STFC). Furthermore, I am indebted to the support of all staff at the European Southern Observatory (ESO), National Radio Astronomy Observatory (NRAO), National Astronomical Observatory of Japan (NAOJ), European Space Agency (ESA), Space Telescope Science Institute (STScI), and National Aeronautics and Space Administration (NASA). Additionally, my gratitude goes out to the developers of all software tools and packages used for this research, for which the relevant acknowledgements are listed at the end of each chapter. In particular, however, I am thankful to the developers of the \program{SciPy} library \citep{Jones2001}, its \program{NumPy} \citep{2011CSE....13b..22V} and \program{Matplotlib} \citep{Hunter2007} modules, the \program{Astropy} package \citep{2013A&A...558A..33A, 2018AJ....156..123A}, \program{ds9} \citep{2003ASPC..295..489J}, \program{topcat} \citep[Tool for OPerations on Catalogues And Tables;][]{2005ASPC..347...29T}, \program{qfitsview} (\url{https://www.mpe.mpg.de/~ott/QFitsView/}), and the \LaTeX\ typesetting system (\url{https://www.latex-project.org}). Finally, it has been a pleasure to connect and work with collaborators and colleagues beyond the Kavli throughout my studies. I am especially grateful to the Instituto de Astronomia, Geof{\'i}sica e Ci{\^e}ncias Atmosf{\'e}ricas at the Universidade de S{\~a}o Paulo for hosting the fantastic \textit{First Light} school which included an unforgettable birthday in the Southern Hemisphere.
    
    Outside of academia, I feel fortunate to have been immersed in Cambridge's kaleidoscopic community, in particular that of my college, Sidney Sussex. I have fond memories of Park Parade (though I do hope the heating is more reliable these days) with my close cycling companion and next-door neighbour Nic and the multi-faceted historians-turned-lawyers Brennan and Chris. The Park Parade core crew went on to begin a new chapter on George Street where through the roll-out of Brexit, I was glad to keep up my Dutch with Jeanine. For entertainment, we could always rely on adventurous travel stories from Kat. Reincarnations of the ``PhD family'' with Michael, Leonie, Amy, and Bekah built a house legacy of lively dinners, indoor mini-golf, and questionable TV shows -- Theo, Ollie, and Ben are off to a decent start so far. The warmth of George Street has been a pillar of strength in uncertain times, in no small part thanks to Heather who has, until recently, tolerated my rowing and cycling chat the longest. While on that subject, I am grateful to be part of two sports clubs that have come to mean much more than that. Despite being told rowing is ``the one thing you can't combine with Part~III'', I am indebted to SSBC where I have made lasting memories both in the boat and on the bank. In particular, I thank my fellow bowsiders Tim, Philipp, and G{\"u}nther for sticking with me through the highs and the lows (maybe let's avoid street brawls in the future). The love-hate relationship I had with cycling in school has decidedly changed to just the former thanks to CUCC, where I think I can now tell apart all the different Jacks, Toms, Matts, and Jameses.
    
    Special thanks goes to all friends back home who have mocked my gradually deteriorating Dutch and occasionally made it across the pond for a visit. In particular, I am glad my experiences mostly consisting of excessive rowing and cycling, and of life as a graduate student, are not entirely lost on my crewmates at Njord. Going back even further in time, it is always a pleasure to reconnect and reminisce with Gerhard, Boyd, Anand, Coen, Sander, and Jelger.
    
    I would like to acknowledge Paul and Marinke who kindly offered their house for a writer's retreat, as well as Darwin who offered me particularly fierce encouragement. It goes without saying I feel deeply privileged for the overwhelming support from all of my family, including my grandparents that I hope look on from the heavens and to whom I dedicate this thesis. Finally, I am deeply grateful to my parents and sister for everything they have done for me. Dankjewel voor al jullie steun Papa, Mama en Caroline.
    
\end{acknowledgements}
