%!TEX root = ../thesis.tex
% ******************************* Thesis Appendix A ****************************
\chapter{Modelling \texorpdfstring{\lymana}{\lymanatext} emission}
\label{ap:Modelling Lya emission}

%\renewcommand\thesection{A.\arabic{section}}
%\setcounter{section}{0}

%\renewcommand\thefigure{A.\arabic{figure}}
%\setcounter{figure}{0}

\section{Model parameters and fitting functions}
\label{ap:Model parameters}

\subsection{Emission processes}

This section contains the fitting functions for the relevant quantities in the formulae for recombination and collisional excitation emissivity (\cref{eq:Recombination emissivity,eq:Collisional excitation emissivity} in \cref{ssec:Lya recombination emission,ssec:Lya collisional excitation emission}), which are repeated here for clarity.

\vspace{1.5ex} \noindent Recombination emissivity (\cref{eq:Recombination emissivity}) is written as
\begin{equation}
    \label{eq:AppRecombination emissivity}
    \epsilon_\text{rec}(T) = f_\text{rec, A/B} (T) \, n_\text{e} \, n_\text{HII} \, \alpha_\text{A/B}(T) \, E_\text{\lya}
    .\end{equation}
\noindent Collisional excitation emissivity (\cref{eq:Collisional excitation emissivity}) is given by
\begin{equation}
    \label{eq:AppCollisional excitation emissivity}
    \epsilon_\text{exc}(T) = \gamma_\text{1s2p} (T) \, n_\text{e} \, n_\text{HI} \, E_\text{\lya}
    .\end{equation}

\subsubsection{Recombination fitting functions}

The underlying equation governing \lya\ emission due to recombination in the IGM is given in \cref{eq:AppRecombination emissivity}. The recombination fraction $f_\text{rec, A/B}$ gives the number of recombinations that ultimately result in the emission of a \lya\ photon. This fraction can be modelled using the relations given in \citet{2008ApJ...672...48C} and \citet{2014PASA...31...40D} and can be summarised as follows:
\begin{equation*}
    f_\text{rec, A/B} = \left\{
    \begin{array}{r}
        \multicolumn{1}{l}{0.41 - 0.165 \log_{10} \left( \frac{T}{10^4 \, \mathrm{K}} \right)} \\
        
        \qquad \qquad \qquad - 0.015 \left( \frac{T}{10^4 \, \mathrm{K}} \right)^{-0.44}, \, \text{case-A} \\
        
        \multicolumn{1}{l}{0.686 - 0.106 \log_{10} \left( \frac{T}{10^4 \, \mathrm{K}} \right)} \\
        
        \qquad \qquad \qquad - 0.009 \left( \frac{T}{10^4 \, \mathrm{K}} \right)^{-0.44}, \, \text{case-B.} \\
    \end{array}
    \right.
\end{equation*}

\noindent The recombination coefficient, $\alpha_\text{A/B}$, is given in the work of \citet{2011piim.book.....D} as follows:
\begin{equation*}
    \alpha_\text{A/B} = \left\{
    \begin{array}{r}
        \multicolumn{1}{l}{4.13 \cdot 10^{-13} \left( \frac{T}{10^4 \, \mathrm{K}} \right)^{-0.7131-0.0115 \log_{10} \left( \frac{T}{10^4 \, \mathrm{K}} \right)} \, \mathrm{cm^{3} \, s^{-1}},} \\
        
        \text{case-A} \\
        
        \multicolumn{1}{l}{2.54 \cdot 10^{-13} \left( \frac{T}{10^4 \, \mathrm{K}} \right)^{-0.8163-0.0208 \log_{10} \left( \frac{T}{10^4 \, \mathrm{K}} \right)} \, \mathrm{cm^{3} \, s^{-1}},} \\
        
        \text{case-B.} \\
    \end{array}
    \right.
\end{equation*}

\subsubsection{Collisional excitation fitting functions}

For collisional excitation, the \lya\ luminosity density is given by \cref{eq:AppCollisional excitation emissivity}. The function $\gamma_\text{1s2p}$ in this formula is given by
\begin{equation}
    \gamma_\text{1s2p} (T) = \Gamma(T) \exp \left( -\frac{E_\text{\lya}}{k_\text{B}T} \right),
\end{equation}

\noindent where $k_\text{B}$ is the Boltzmann constant. The function $\Gamma(T)$ is characterised in \citet{1990MNRAS.242..692S} and \citet{1991ApJ...380..302S} as follows:
\begin{equation}
    \label{eq:Gamma}
    \Gamma (T) = \exp \left( \sum_{i=0}^{5} c_i \left( \ln T \right)^i \right),
\end{equation}

\noindent where the coefficients $c_i$ found by \citet{1990MNRAS.242..692S} and \citet{1991ApJ...380..302S} are dependent on the temperature regime (shown in \cref{tab:Coefficients}). As noted in \cref{sssec:Collisional excitation emissivity}, the rates are not identical to those applied in the cosmological hydrodynamical simulation, but in the relevant temperature regime deviate so little that the \lya\ emission would not be appreciably changed.
\begin{table}
    \centering
    \caption[Coefficients $c_i$ in \cref{eq:Gamma}]{
        Coefficients $c_i$ in \cref{eq:Gamma} and their corresponding temperature regimes.
    }
    \begin{tabular}{c c l l l}
        
        & & \textbf{Regime 1}   & \textbf{Regime 2}     & \textbf{Regime 3} \\
        
        $c_0$ & & $-1.630155 \cdot 10^{2}$      & $5.279996 \cdot 10^{2}$ & $-2.8133632 \cdot 10^{3}$ \\
        
        $c_1$ & & $8.795711 \cdot 10^{1}$       & $-1.939399 \cdot 10^{2}$ & $8.1509685 \cdot 10^{2}$ \\
        
        $c_2$ & & $-2.057117 \cdot 10^{1}$      & $2.718982 \cdot 10^{1}$ & $-9.4418414 \cdot 10^{1}$ \\
        
        $c_3$ & & $2.359573$                            & $-1.883399$                           & $5.4280565$ \\
        
        $c_4$ & & $-1.339059 \cdot 10^{-1}$     & $6.462462 \cdot 10^{-2}$        & $-1.5467120 \cdot 10^{-1}$ \\
        
        $c_5$ & & $3.021507 \cdot 10^{-3}$      & $-8.811076 \cdot 10^{-4}$        & $1.7439112 \cdot 10^{-3}$ \\
    \end{tabular}
    
    \vspace{0.5ex}
    
    \begin{tabular}{c c}
        
        \textbf{Regimes}        & \textbf{Temperature values} \\
        
        Regime 1        & $2 \cdot 10^3 \, \mathrm{K} \leq T < 6 \cdot 10^4 \, \mathrm{K}$ \\
        
        Regime 2        & $6 \cdot 10^4 \, \mathrm{K} \leq T < 6 \cdot 10^6 \, \mathrm{K}$ \\
        
        Regime 3        & $6 \cdot 10^6 \, \mathrm{K} \leq T \leq 1 \cdot 10^8 \, \mathrm{K}$ \\
    \end{tabular}
    \label{tab:Coefficients}
\end{table}

\begin{figure}
    \centering
    \includegraphics[width=\columnwidth]{"Plots/Chapter1/Optical_depths"}
    \caption[2D histogram of \lya\ optical depth $\tau$ and overdensity $\rho/\bar{\rho}$ at $z=4.8$]{Two-dimensional density histogram for each of $2048$ pixels in spectra along $5000$ (randomly selected) lines of sight at $z=4.8$, as a function of the \lya\ optical depth $\tau$ and overdensity $\rho/\bar{\rho}$ in the sightline; the two parameters are measured at line centre, where the optical depth was divided by $2$ just to account for the hydrogen between the source and the observer (see text).}
    \label{fig:2D scatter}
\end{figure}
\begin{figure*}
    \centering
    \includegraphics[width=\linewidth]{"Plots/Chapter1/Column_density"}
    \caption[Observed column density at $z=4.8$]
    {Simulated column density of neutral hydrogen, $N_\text{HI}$, in a simulation snapshot at $z=4.8$. The same regions as in \cref{fig:4nsobs_ov} are shown. Moreover, the same density thresholds used for the collisional excitation component are applied, that is only gas below half the critical self-shielding density is shown, meaning this is the column density that would correspond to a narrowband image of the low-density gas with $\Delta \lambda_\text{obs} = 3.75 \, \text{\AA}$ ($\ssim 1.19 \, h^{-1} \, \mathrm{cMpc}$). Panel~\textbf{a} shows an overview of part of the simulation snapshot that corresponds to region~1 in \cref{fig:SB}. This is centred on the same comoving coordinates both spatially and spectrally, but now less extended in wavelength range. This panel shows a region of $8 \times 8 \, h^{-2} \, \mathrm{cMpc}^2$ ($5.2 \times 5.2 \, \mathrm{arcmin}^2$) on a pixel grid of $1024 \times 1024$. Panels~\textbf{b}-\textbf{d} show column density maps of neutral hydrogen the size of the MUSE FOV consisting of $300 \times 300$ pixels. The areas covered by these maps are indicated by the black squares in the overview panel~a. In the bottom left corner of each panel, two different measures of the overdensity of the region are shown (see \cref{ssec:Simulated observations} for more details). In panels~b-d, halos with halo mass of $M_\mathrm{h} > 10^{9.5} \, \mathrm{M_\odot}$ are shown as circles, their size indicating their projected virial radius (see \cref{ssec:Simulated observations}). The most massive halo in each panel is annotated.}
    \label{fig:NHI}
\end{figure*}
\begin{figure*}
    \centering
    \includegraphics[width=\linewidth]{"Plots/Chapter1/Redshift_SB_evolution"}
    \caption[Observed \lya\ surface brightness at different redshifts]
    {\lya\ SB for a combination of recombination emission (of all gas in the simulation) below the mirror limit and collisional excitation of gas below half the critical self-shielding density at different redshifts. Both the mirror limit and the critical self-shielding density evolve as a function of redshift (see \cref{fig:UVB_limits}). The panels show snapshots at redshifts of $z=6.00$, $z=5.58$, $z=4.49$, $z=4.00$, $z=3.60$, $z=3.20$ for a narrowband with $\Delta \lambda_\text{obs} = 3.75 \, \text{\AA}$; at $z=5.76$; this corresponds to $\ssim 1.19 \, h^{-1} \, \mathrm{cMpc}$, but this again changes with redshift. The panels all display a pixel grid of $300 \times 300$ and the angular size of the MUSE FOV ($1 \times 1 \, \mathrm{arcmin}^2$), which translates to different physical sizes at each corresponding redshift. The regions are all centred at the same comoving transverse coordinates as panel~d in \cref{fig:4nsobs_ov} and \cref{fig:4nsobs_ov150}; however, the narrowband centre (the coordinate along the line of sight) has been chosen to coincide with the most massive halo in each panel to ensure the entire filament is captured in each panel. The two numbers in the bottom left corner show the same two different measures of the overdensity of the region, $\Delta_\mathrm{baryon}$ and $\Delta_\mathrm{halo}$ (see \cref{sssec:Cosmic variance and narrowband widths} for more details). Halos with halo mass of $M_\mathrm{h} > 10^{9.5} \, \mathrm{M_\odot}$ are shown as circles, their size indicating their projected virial radius. The most massive halo in each panel is annotated. The scale varies between different panels since the angular size is kept constant across all redshifts.}
    \label{fig:4nsobs_mos}
\end{figure*}

\section{\texorpdfstring{\lya}{\lyatext} optical depth}
\label{ap:Lya optical depth}

This work does not contain treatment of \lya\ line radiative transfer effects (\cref{sssec:Radiative transfer effects}). For our purposes, the treatment without radiative transfer gives us valuable insights into the lower-density IGM filaments on large, cosmological scales without having to resort to implementing computationally expensive radiative transfer methods that are difficult to accurately model, for example because the effects of dust are poorly constrained.

In \cref{fig:2D scatter}, a two-dimensional density histogram for each of $2048$ pixels in mock \lya\ absorption spectra along $5000$ lines of sight at $z=4.8$ is shown as a function of both the \lya\ optical depth $\tau$ and overdensity $\rho/\bar{\rho}$ in the relevant pixel. These spectra are extracted on the fly at redshift intervals $\Delta z = 0.1$ and are constructed from the gas density and neutral fraction, temperature, and peculiar velocity of neutral hydrogen along these lines of sight \citep[for details, see][ where they are studied in the context of the \lya\ forest]{2017MNRAS.464..897B}. The peculiar velocity of the gas in a given pixel has been used to translate its position to redshift space where optical depth is determined. Therefore both density and optical depth are effectively measured at line centre. The optical depth was divided by a factor of $2$ to account for the fact that on average only half of the matter is in between the source and the observer; the other half is located behind the source.\footnote{We note that the division by $2$ is necessary as the \lya\ optical depths were originally extracted to study \lya\ forest absorption in the spectra of background sources in which case all the gas that affects a pixel in redshift space is in front of the source in real space.} From this figure, it is clear that at mean density optical depths of order $10$ are reached, indicating that radiative transfer has an effect on most regions. However, effectively this plot still shows an overestimated measure of optical depth. Since it uses a measure of optical depth at line centre, this does not mean that physically no \lya\ emission is detected in the optically thick regime ($\tau > 1$). Many \lya\ photons may be able to escape because an initial scattering not only changes the direction of propagation of photons, but also shifts their frequency and the optical depth decreases quickly when moving away from line centre. An example of this effect is the \lya\ radiation from galaxies, where densities are high enough to have optical depths of the order of $10^6$, but escape away from line centre is still possible. The optical depth thus mostly informs the expected degree of scattering, that is spatial and spectral broadening of the line profile.

Additionally, the neutral hydrogen (\ion{H}{I}) column density at $z=4.8$ is shown in \cref{fig:NHI} for precisely the same simulation region (and density limits used for collisional excitation) as in \cref{fig:4nsobs_ov}, with the same narrowband width of $\Delta \lambda_\text{obs} = 3.75 \, \text{\AA}$ (equivalent to $\ssim 1.19 \, h^{-1} \, \mathrm{cMpc}$), and pixel grids of pixel grid of $1024 \times 1024$ (panel~a) and $300 \times 300$ (panels~b-d). The overview map (panel~a) shows that all areas have column densities of at least $N_\text{HI} \sim 10^{15} \, \mathrm{cm^{-2}}$. The most extreme features of the low-density gas show column densities of $10^{17}$-$10^{18} \, \mathrm{cm^{-2}}$, which is the range of Lyman-limit systems. Except for the self-shielding prescription, simulations that are very similar to that used in this work are found to match observational \ion{H}{I} column density distributions well at lower redshifts, where data are more abundant, at least up to $N_\text{HI} \sim 3 \cdot 10^{16} \, \mathrm{cm^{-2}}$, where self-shielding is expected to have a negligible effect \citep{2017MNRAS.464..897B}. At higher column densities, the self-shielding prescription that we use \citep{2013MNRAS.430.2427R} was calibrated to yield realistic column density distributions. At the highest column densities, our simulation will certainly be affected by our simplistic galaxy formation model. These high densities are, however, not the focus of this study.

As with \cref{fig:2D scatter}, it has to be taken into account that this is the column density projected for the entire narrowband. Emitting structures seen within this slice always lie between the boundaries of this region. Therefore part of the column density that is projected may be behind the emitting region, as seen from the observer's perspective. This means that, on average, the actual values of column densities photons travels through is about half of what is displayed.

As discussed in \cref{sssec:Radiative transfer effects}, it is expected that the precise way in which these scattering processes affect the perceived SB images are the result of a competition between two underlying effects. One possibility is that the photons emerging from the filamentary structure might be spread out, causing the signal to become fainter. The second possibility is that the filament signal might be enhanced by \lya\ radiation coming from nearby dense structures (where additional radiation is likely to be produced in galaxies) that is scattered in the filament, thereby causing the filaments to appear brighter. As mentioned, similar simulations including radiative transfer show a mixture of these two effects, where the SB of filaments generally is not affected much or even boosted \citetext{private communication, Weinberger, 2019}.

\section{Redshift evolution}
\label{ap:Redshift evolution}

The region extensively discussed in \cref{ssec:Simulated observations}, shown in panel~d of \cref{fig:4nsobs_ov} and all panels in \ref{fig:4nsobs_ov150}, is shown at different redshifts in \cref{fig:4nsobs_mos}, again showing the combination of recombination emission of all gas in the simulation below the mirror limit and collisional excitation of gas below half the critical self-shielding density. The panels shown are centred at the same transverse comoving coordinates as panel~d in \cref{fig:4nsobs_ov} and all panels in \ref{fig:4nsobs_ov150}, but the narrowband centre (the coordinate along the line of sight) is now chosen to coincide with the most massive halo in each panel to ensure the same structure is captured in each panel. Each panel covers the angular size of the MUSE FOV, the physical extent of which varies at different redshifts.

Following the redshift evolution from high to low (going from panel~a to panel~f), we note that the comoving size of the observed region shrinks roughly from $\ssim 1.5 \times 1.5 \, h^{-2} \, \mathrm{cMpc}^2$ to just over $\ssim 1 \times 1 \, h^{-2} \, \mathrm{cMpc}^2$ since the angular size of the FOV is kept fixed at $1 \times 1 \, \mathrm{arcmin}^2$. The appearance of new massive ($M_\mathrm{h} > 10^{9.5} \, \mathrm{M_\odot}$) halos and their evolution in relative movement and mass accretion, indicated by the increase in their virial radii, can also be traced between the different panels. Panel~e, at $z=3.60$, has the same redshift as shown in the bottom two panels of \cref{fig:4nsobs_ov150}.

With these conservative limits that exclude emission from the dense (and complicated) central regions of halos, \lya\ emission appears brighter at low redshift, where the mirror limit is less affected by SB dimming and self-shielding effects only start to play a role at higher overdensities, as discussed in \cref{sssec:Sensitivity analysis}. Panels~a and b appear particularly homogeneous as large portions are impacted by the mirror limit ($24.6\%$ and $22.1\%$ of pixels exceeding the mirror limit). We note that at low redshift, on the other hand, there is less low-density gas that is luminous in \lya, especially within the large, central halo. The gas there is likely denser and hotter and thus less effective at emitting \lya\ radiation, at least within the low-density regime that we are considering (cf. \cref{fig:z_evolution_lum,fig:Luminosity phase space} and their discussion in the text).