%!TEX root = ../thesis.tex
% ******************************* Thesis Appendix A ****************************
\chapter{Photoionisation models}
\label{app:Photoionisation_model}

The following discussion considers an \HII\ region as a spherically-symmetric, steady-state system (initially ignoring helium for simplicity). An ionisation equilibrium state requires that the rate of ionisations equals the rate of recombinations. The rate of ionisations within a certain radius $r$ can simply be equated to $\dot{N}_\text{ion}$, the number of ionising photons emitted by the star per unit time. The recombination rate per unit volume is $\alpha_\text{B} \, n_\text{e} \, n_\text{\HII}$, with $\alpha_\text{B}$ being the recombination coefficient (in case B, where only recombinations into an excited state are considered; see \cref{chPsssec:Recombination emissivity} for further details) and $n_\text{e}$ and $n_\text{\HII}$ the number densities of electrons and ionised hydrogen.

A further simplification of the model is to assume hydrogen has constant number density, $n_\text{H}$, and is entirely ionised in the central region (i.e. $n_\text{e} = n_\text{\HII} = n_\text{H}$). This defines the \citeauthor{1939ApJ....89..526S} radius $R_\text{S}$ given in \cref{chIeq:Photoionisation_equilibrium}, enclosing the volume $\frac{4}{3} \pi R_\text{S}^3$ for which ionisation equilibrium holds. Detailed calculations show that indeed hydrogen is ionised to a high degree within the \citeauthor{1939ApJ....89..526S} radius, as will be illustrated next with a numerical photoionisation code.
\begin{figure}
    \centering
    \includegraphics[width=0.95\linewidth]{"Figs/Cloudy_PDR_structure"}
    \caption[Structure of an \HII\ region transitioning into a PDR]
    {Structure of an \HII\ region transitioning into a PDR. Depth into the plane-parallel nebula is measured in dust attenuation, $A_V$ (see \cref{chIssec:Effects_of_dust}; bottom axis), and physical distance measured from the illuminated face of the cloud, the depth $d$ (top axis). The top panel shows the number densities of hydrogen and electrons (solid lines) and the electron temperature (dashed line; see \cref{chIssec:Nebular_emission_and_emission-line_diagnostics}). In the second panel, the fraction of ionised hydrogen and helium (singly and doubly) are shown. The third panel shows the same for carbon and oxygen (up to the triply ionised species, $\mathrm{C^{3+}}$ and $\mathrm{O^{3+}}$). In the bottom panel, normalised fractions of the most abundant molecules ($\mathrm{H_2}$, $\mathrm{H_2 O}$, and $\mathrm{CO}$) are shown.}
    \label{appIfig:Cloudy_PDR_structure}
\end{figure}

Such codes can carry out complex ionisation equilibrium calculations, tracking multiple ionisation states of all elements present in the cloud. As will be discussed in \cref{chDssec:Discussion:OIII/CII_photoionisation_models,chDssec:Discussion:OIII/CII_theoretical_insights}, \program{Cloudy} \citep[e.g.][]{2017RMxAA..53..385F} is one such code able to construct one-dimensional, radiative-transfer models of a plane-parallel nebula, given a full specification of the incident radiation field and chemical composition of the gas. As an example, the inner structure of a $z = 7$ \HII\ region (transitioning into a zone of neutral, molecular gas in the form of a photodissociation region or PDR; cf. \cref{ch:Dual_constraints_with_ALMA}) simulated by \program{Cloudy} is shown in \cref{appIfig:Cloudy_PDR_structure}. These models will be described in detail in \cref{ch:Dual_constraints_with_ALMA}, but I give a brief summary here.

The incident radiation field of a stellar population, formed in a single burst of star formation with an age of $1 \, \mathrm{Myr}$, is generated by \program{bpass} v2.1 stellar population synthesis models \citep[including binary stars;][]{2017PASA...34...58E} under a \citeauthor{1955ApJ...121..161S} IMF, ranging in stellar mass from $1 \, \mathrm{M_\odot}$ to $100 \, \mathrm{M_\odot}$.\footnote{Note that the model is slightly different from a classical \HII\ region described above, given the plane-parallel geometry and the fact that the central ionising source is not a single star, but the integrated spectrum of a stellar population.} The spectrum is normalised by the ionisation parameter, as defined in \cref{chIeq:Ionisation_parameter}, in this case set to $\log_{10} U = -2.5$. Note, however, that the definitions differ slightly in plane-parallel geometry: in contrast to a spherical geometry, where $F_\text{ion} (r) \propto 1/r^2$ (in the absence of absorption), in the plane-parallel case the flux of ionising photons does not change with $d$, the depth into the cloud.\footnote{In practice, \program{Cloudy} takes an ionisation parameter to set $F_\text{ion} = q_\text{ion} n_\text{H}$ at the illuminated face of the cloud.} Hence, the effective ``\citeauthor{1939ApJ....89..526S} depth'' -- defined analogously to \cref{chIeq:Photoionisation_equilibrium} as the extent within which the plane-parallel nebula would be fully ionised -- is $d_\text{S} = q_\text{ion}/\alpha_\text{B} \, n_\text{H}$, a factor three smaller than $R_\text{S} = 3q_\text{ion}/\alpha_\text{B} \, n_\text{H}$ that follows from combining \cref{chIeq:Photoionisation_equilibrium,chIeq:Ionisation_parameter}.