%!TEX root = ../thesis.tex
%*******************************************************************************
%******************************* Introduction **********************************
%*******************************************************************************

\chapter*{Introduction}
\label{ch:Introduction}
\addcontentsline{toc}{chapter}{Introduction}
\chaptermark{Introduction}{}

\renewcommand\thefigure{I.\arabic{figure}}
\setcounter{figure}{0}

\setlength{\epigraphwidth}{0.8\textwidth}
\epigraph{\textit{``We are just an advanced breed of monkeys on a minor planet of a very average star. But we can understand the Universe. That makes us something very special.''}}{Stephen Hawking}

Arguably the oldest and most prominent question in the history of humanity, captivating not only many philosophers and scientists, but really all humankind -- both religious and irreligious, wise and unsophisticated, civilised and uncivilised -- is

\renewcommand{\epigraphrule}{0pt}
\setlength{\epigraphwidth}{\textwidth}
\setlength{\beforeepigraphskip}{1 ex}
\setlength{\afterepigraphskip}{-2 ex}
\renewcommand{\epigraphsize}{\large}

\epigraph{\flushleft {\textit{``Where do we come from?''}}}{}

\noindent Among other related questions, this might even be the one that drove humans to engage in an endeavour that we presently categorise as \textit{science}, and for many, it probably still is the profound motivation to continue investigating nature and the astonishing phenomena it presents us. Although seemingly simple, it has proven to be one of the most difficult scientific questions to answer -- if it even can be identified as a proper scientific question, given its broad and universal character.

\par Even so, many have made great progress on finding the answer to this mystery. There are numerous possible approaches to achieving this, but to scientifically divide up this general question into specific and rational problems seems to have proven the most successful and fruitful way.

\section*{The Universe}

Through astronomical observations, we have learned a tremendous amount about the \textit{cosmos}, or everything that exists around us. We have, for example, concluded that we reside in the Solar System, and we inhabit one of eight planets orbiting the Sun, a very ordinary star, which itself is part of the Milky Way, an immense collection of stars, mostly very similar to our Sun.

\par The Milky Way is, again, quite an average galaxy, of which extremely many exist. Astronomers observe them in different shapes and sizes, and even at different times in the past, which is possible due to the finite velocity that light travels with. For this reason, they are able to study their evolution and thereby the evolution of the Universe as a whole. Moreover, this allows for the investigation of the origins of the cosmos by reversing the physical processes that we have identified and can mathematically describe.
\begin{figure}[b!]
	\centering
	\includegraphics[width=0.8\linewidth]{"Images/HUDF"}
	\caption[Hubble Ultra-Deep Field]{Hubble Ultra-Deep Field (HUDF) image, showing a range of distant galaxies in the wavelength range of near-infrared to ultraviolet. Credit: NASA/ESA/Hubble}
	\label{fig:HUDF}
\end{figure}

\par However, constrained by observational limits posed to telescopes, mainly due to the astronomical sizes the Universe assumes (the galaxy closest to the Milky Way, Andromeda, is approximately \num{60000000000000} times more distant from Earth than the Moon is), many questions still remain to be answered.

\noindent Although our understanding of the Universe at this point is far-reaching, many questions about it still remain unanswered. In cosmology, a very robust model of the Universe has been developed over the years, built upon on the theory of General Relativity, and called after its primary components: the Lambda Cold Dark Matter ($\Lambda$CDM) cosmological model. This theory has proven extremely successful in explaining a broad range of distinct phenomena, being able to predict the outcome of experiments with very high precision, after having been provided with only a handful of parameters. However, the present model still is undecided about a number of controversies, and is known to have some issues -- therefore, it has to be tested rigorously, and possibly amended\footnote{Or, according to some, even replaced\dots} to match the observational evidence that is continuously supplied or improved upon.

\par Specifically, the question of how stars precisely are formed in the quantities that eventually together form galaxies that are observed today, assembling with other galaxies to form massive galaxy clusters, that are again part of an even bigger structure -- the very largest of structures that we observe in the Universe today, termed the {\textit{cosmic web}}\index{cosmic web} -- is a ubiquitous topic in present cosmological, and indeed astronomical research. The general framework behind the process of the formation of this structure, in which galaxy formation is a vital element, is understood in the $\Lambda$CDM model. Still, a variety of subtly different versions of these models have to be tested in order to decide which can explain best how galaxies and other structures have developed, and thereby discover more clearly how the Universe evolved as a whole. The cosmic web is thought to play a major role in this evolution.

\section*{Cosmic web}

On the very large scales -- that is, at distances of well over \num{10}~Mpc -- the Universe is known to exhibit a filamentary structure: this is known as the {cosmic web}\index{cosmic web}. This web contains most of the matter present in the Universe, leaving large voids with extremely low densities in between the filaments. At the intersection of filamentary structures, galaxies cluster together in groups. Baryonic matter can partly be traced down to reside in stars and gas in these galaxies, whose distribution follows the filamentary structure -- surprisingly, however, most of the visible matter in the Universe is actually thought to be located in the {\textit{intergalactic medium}}\index{intergalactic medium} ({IGM}\index{IGM|see {intergalactic medium}}). There, it manifests as a cold, dilute gas; hydrogen is by far the largest constituent of the IGM, while there is little metal enrichment.

\section*{Main research goals}

This report investigates\dots
