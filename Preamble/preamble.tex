% ******************************************************************************
% ****************************** Custom Margin *********************************

% Add `custommargin' in the document class options to use this section
% Set {innerside margin / outerside margin / topmargin / bottom margin}  and
% other page dimensions
\ifsetCustomMargin
% ******************************* Joris' edit **********************************
  %\usepackage[left=37mm,right=30mm,top=35mm,bottom=30mm]{geometry}
  \usepackage[left=15mm, right=15mm, top=20mm, bottom=20mm]{geometry}
  \setFancyHdr % To apply fancy header after geometry package is loaded
\fi

% Add spaces between paragraphs
%\setlength{\parskip}{0.5em}
% Ragged bottom avoids extra whitespaces between paragraphs
\raggedbottom
% To remove the excess top spacing for enumeration, list and description
%\usepackage{enumitem}
%\setlist[enumerate,itemize,description]{topsep=0em}

% *****************************************************************************
% ******************* Fonts (like different typewriter fonts etc.)*************

% Add `customfont' in the document class option to use this section

\ifsetCustomFont
  % Set your custom font here and use `customfont' in options. Leave empty to
  % load computer modern font (default LaTeX font).
%  \usepackage{mathptmx}
%  \usepackage{fourier}
  \usepackage[scaled]{helvet}
  
  % Sets the sans-serif family as the default family (required to use a sans-serif font such as Utopia in the Fourier package or Helvetica)
%  \renewcommand{\familydefault}{\sfdefault}

  % For use with XeLaTeX
  %  \setmainfont[
  %    Path              = ./libertine/opentype/,
  %    Extension         = .otf,
  %    UprightFont = LinLibertine_R,
  %    BoldFont = LinLibertine_RZ, % Linux Libertine O Regular Semibold
  %    ItalicFont = LinLibertine_RI,
  %    BoldItalicFont = LinLibertine_RZI, % Linux Libertine O Regular Semibold Italic
  %  ]
  %  {libertine}
  % load font from system font
%  \newfontfamily\libertinesystemfont{Linux Libertine O}
\fi

% *****************************************************************************
% **************************** Custom Packages ********************************
% ******************************* Joris' edit **********************************

\newcommand{\program}{\textsc}
\newcommand{\ssim}{\sim \!}
\def\app#1#2{%
    \mathrel{%
        \setbox0=\hbox{$#1\sim$}%
        \setbox2=\hbox{%
            \rlap{\hbox{$#1\propto$}}%
            \lower1.1\ht0\box0%
        }%
        \raise0.25\ht2\box2%
    }%
}
\def\appropto{\mathpalette\app\relax}

\definecolor{COS2987}{rgb}{0.8941176470588236, 0.10196078431372549, 0.10980392156862745}
\definecolor{COS3018}{rgb}{0.21568627450980393, 0.49411764705882355, 0.7215686274509804}
\definecolor{UVIS001}{rgb}{0.30196078431372547, 0.6862745098039216, 0.2901960784313726}
\definecolor{UVIS007}{rgb}{0.596078431372549, 0.3058823529411765, 0.6392156862745098}
\definecolor{UVIS019}{rgb}{1.0, 0.4980392156862745, 0.0}
\definecolor{ALMAsky}{RGB}{231, 231, 221}

\newcommand{\Isample}{\citeauthor{2016Natur.529..178I} sample}

\newcommand{\LCDM}{\ensuremath{\mathrm{\Lambda CDM}}}

\newcommand{\lymana}{{Lyman-\ensuremath{\upalpha}}}
\newcommand{\lymanatext}{Lyman-α}
\newcommand{\lya}{{Ly\ensuremath{\upalpha}}}
\newcommand{\lyatext}{Lyα}

\newcommand{\Angstrom}{\text{\AA}}

\newcommand{\arcsec}{\hbox{$^{\prime\prime}$}}
\newcommand{\ion}[2]{\textup{#1\,\textsc{\lowercase{#2}}}}

\newcommand{\HII}{{\ion{H}{II}}}
\newcommand{\CIV}{{\ion{C}{IV}}}
\newcommand{\HeII}{{\ion{He}{II}}}
\newcommand{\OIIIf}{{[\ion{O}{III}]}}
\newcommand{\OIIIs}{{\ion{O}{III}]}}
\newcommand{\CII}{{[\ion{C}{II}]}}
\newcommand{\CIII}{{\ion{C}{III}}}
\newcommand{\CIIIf}{{[\ion{C}{III}]}}
\newcommand{\CIIIs}{{\ion{C}{III}]}}
\newcommand{\MgII}{{\ion{Mg}{II}}}
\newcommand{\OII}{{[\ion{O}{II}]}}
\newcommand{\NeIII}{{[\ion{Ne}{III}]}}

\newcommand{\Halpha}{{\ensuremath{\mathrm{H\upalpha}}}}
\newcommand{\Hbeta}{{\ensuremath{\mathrm{H\upbeta}}}}
\newcommand{\NII}{{[\ion{N}{II}]}}
\newcommand{\SII}{{[\ion{S}{II}]}}
\newcommand{\OI}{{[\ion{O}{I}]}}

\newcommand{\CIILam}{{\CII\ 158 $\upmu\mathrm{{m}}$}}
\newcommand{\OIIILam}{{\OIIIf\ 88 $\upmu\mathrm{{m}}$}}
\newcommand{\NIILam}{{\NII\ 205 $\upmu\mathrm{{m}}$}}

\usepackage{commath}
\usepackage{upgreek}
\usepackage[greek, english]{babel}

\usepackage{ae,aecompl}
\usepackage{color, soul}
\usepackage[dvipsnames]{xcolor}
\usepackage{csvsimple}

\usepackage[leftmargin=10pt, rightmargin=10pt, skipabove=5pt, skipbelow=5pt, innerleftmargin=5pt, innerrightmargin=5pt, innertopmargin=10pt, innerbottommargin=10pt, linewidth=0pt, innerlinewidth=0pt, middlelinewidth=0pt, outerlinewidth=0pt, roundcorner=0pt]{mdframed}
\usepackage{wrapfig}

\usepackage{epigraph}
\usepackage{lettrine}
\usepackage[super]{nth}
\usepackage{metalogo}

\usepackage[pages=some, placement=top]{background}

\newcommand{\CurrentTitleColor}{\color{black}}
\usepackage{titlesec}
\newcommand{\PreContentTitleFormat}{\titleformat{\chapter}[display]{\CurrentTitleColor\scshape\huge}
	{\CurrentTitleColor\Large\filleft{\chaptertitlename} \Huge\thechapter}
	{0ex}{}
%	[\vspace{1ex}\titlerule]
}
\newcommand{\ContentTitleFormat}{\titleformat{\chapter}[display]{\CurrentTitleColor\scshape\huge}
	{\CurrentTitleColor\Large\filleft{\chaptertitlename} \Huge\thechapter}
    {0ex}{\titlerule\vspace{1ex}\filright}
%	[\vspace{1ex}\titlerule]
}
\newcommand{\PostContentTitleFormat}{\PreContentTitleFormat}
\PreContentTitleFormat

% ************************* Algorithms and Pseudocode **************************

%\usepackage{algpseudocode}

% ********************Captions and Hyperreferencing / URL **********************

% Captions: This makes captions of figures use a boldfaced small font.
%\usepackage[small,bf]{caption}

% ******************************* Joris' edit **********************************
%\usepackage[labelsep=space,tableposition=top]{caption}
\usepackage[small, labelfont=bf, labelsep=space, tableposition=top]{caption}
%\renewcommand{\figurename}{Fig.} %to support older versions of captions.sty
\renewcommand{\figurename}{Figure}


% *************************** Graphics and figures *****************************

%\usepackage{rotating}
%\usepackage{wrapfig}

% Uncomment the following two lines to force Latex to place the figure.
% Use [H] when including graphics. Note 'H' instead of 'h'
\usepackage{float}
\restylefloat{figure}

% Subcaption package is also available in the sty folder you can use that by
% uncommenting the following line
% This is for people stuck with older versions of texlive
%\usepackage{sty/caption/subcaption}
\usepackage{subcaption}

% ********************************** Tables ************************************
\usepackage{booktabs} % For professional looking tables
\usepackage{multirow}

%\usepackage{multicol}
%\usepackage{longtable}
%\usepackage{tabularx}


% *********************************** SI Units *********************************
\usepackage{siunitx} % use this package module for SI units


% ******************************* Line Spacing *********************************

% Choose linespacing as appropriate. Default is one-half line spacing as per the
% University guidelines

% \doublespacing
\onehalfspacing
%\singlespacing


% ************************ Formatting / Footnote *******************************

% Don't break enumeration (etc.) across pages in an ugly manner (default 10000)
%\clubpenalty=500
%\widowpenalty=500

%\usepackage[perpage]{footmisc} %Range of footnote options
\usepackage{fnpct}
\usepackage{scrextend}

\deffootnote{2em}{1em}{\thefootnotemark\hspace{1em}}
\renewcommand*\thempfootnote{\arabic{mpfootnote}}

% ******************************* Joris' edit **********************************
% (Custom: footnotes in tables)
\usepackage{footnote}
\makesavenoteenv{tabular}
\makesavenoteenv{table}


% *****************************************************************************
% *************************** Bibliography  and References ********************

% ******************************* Joris' edit **********************************
% Allow "Thomas van Noord" and "Simon de Laguarde" and alike to be sorted by "N" and "L" etc. in the bibliography. Write the name in the bibliography as "author = {{\VAN{Noord}{Van}{van} Noord}, Thomas}"
\DeclareRobustCommand{\VAN}[3]{#2}
\let\VANthebibliography\thebibliography
\def\thebibliography{\DeclareRobustCommand{\VAN}[3]{##3}\VANthebibliography}

%\usepackage{cleveref} %Referencing without need to explicitly state fig/table
\usepackage[capitalise]{cleveref} %noabbrev
\crefname{figure}{Figure}{Figures}
%\crefname{section}{Sect.}{Sects.}
\crefname{equation}{equation}{equations}

% Ensure appendices are referred to as Appendix A, B, etc.
\usepackage{xpatch}
\xapptocmd\appendices{%
    \crefalias{chapter}{appendix}%
}{}{\PatchFailed}

% Add `custombib' in the document class option to use this section
\ifuseCustomBib
   \usepackage[authoryear]{natbib} % CustomBib
   %\usepackage[comma, sort&compress, authoryear]{natbib} % CustomBib
   %\usepackage[comma, sort&compress, numbers]{natbib} % CustomBib
   
   %------------- Joris' edit (taken from MNRAS template) ----------------
   
   % Standard journal abbreviations
   % Mostly as used by ADS, with a few additions for journals where MNRAS does not
   % follow normal IAU style.
   
   \newcommand\aap{A\&A}                % Astronomy and Astrophysics
   \let\astap=\aap                          % alternative shortcut
   \newcommand\aapr{A\&ARv}             % Astronomy and Astrophysics Review (the)
   \newcommand\aaps{A\&AS}              % Astronomy and Astrophysics Supplement Series
   \newcommand\actaa{Acta Astron.}      % Acta Astronomica
   \newcommand\afz{Afz}                 % Astrofizika
   \newcommand\aj{AJ}                   % Astronomical Journal (the)
   \newcommand\ao{Appl. Opt.}           % Applied Optics
   \let\applopt=\ao                         % alternative shortcut
   \newcommand\aplett{Astrophys.~Lett.} % Astrophysics Letters
   \newcommand\apj{ApJ}                 % Astrophysical Journal
   \newcommand\apjl{ApJ}                % Astrophysical Journal, Letters
   \let\apjlett=\apjl                       % alternative shortcut
   \newcommand\apjs{ApJS}               % Astrophysical Journal, Supplement
   \let\apjsupp=\apjs                       % alternative shortcut
   % The following journal does not appear to exist! Disabled.
   %\newcommand\apspr{Astrophys.~Space~Phys.~Res.} % Astrophysics Space Physics Research
   \newcommand\apss{Ap\&SS}             % Astrophysics and Space Science
   \newcommand\araa{ARA\&A}             % Annual Review of Astronomy and Astrophysics
   \newcommand\arep{Astron. Rep.}       % Astronomy Reports
   \newcommand\aspc{ASP Conf. Ser.}     % ASP Conference Series
   \newcommand\azh{Azh}                 % Astronomicheskii Zhurnal
   \newcommand\baas{BAAS}               % Bulletin of the American Astronomical Society
   \newcommand\bac{Bull. Astron. Inst. Czechoslovakia} % Bulletin of the Astronomical Institutes of Czechoslovakia 
   \newcommand\bain{Bull. Astron. Inst. Netherlands} % Bulletin Astronomical Institute of the Netherlands
   \newcommand\caa{Chinese Astron. Astrophys.} % Chinese Astronomy and Astrophysics
   \newcommand\cjaa{Chinese J.~Astron. Astrophys.} % Chinese Journal of Astronomy and Astrophysics
   \newcommand\fcp{Fundamentals Cosmic Phys.}  % Fundamentals of Cosmic Physics
   \newcommand\gca{Geochimica Cosmochimica Acta}   % Geochimica Cosmochimica Acta
   \newcommand\grl{Geophys. Res. Lett.} % Geophysics Research Letters
   \newcommand\iaucirc{IAU~Circ.}       % IAU Cirulars
   \newcommand\icarus{Icarus}           % Icarus
   \newcommand\japa{J.~Astrophys. Astron.} % Journal of Astrophysics and Astronomy
   \newcommand\jcap{J.~Cosmology Astropart. Phys.} % Journal of Cosmology and Astroparticle Physics
   \newcommand\jcp{J.~Chem.~Phys.}      % Journal of Chemical Physics
   \newcommand\jgr{J.~Geophys.~Res.}    % Journal of Geophysics Research
   \newcommand\jqsrt{J.~Quant. Spectrosc. Radiative Transfer} % Journal of Quantitiative Spectroscopy and Radiative Transfer
   \newcommand\jrasc{J.~R.~Astron. Soc. Canada} % Journal of the RAS of Canada
   \newcommand\memras{Mem.~RAS}         % Memoirs of the RAS
   \newcommand\memsai{Mem. Soc. Astron. Italiana} % Memoire della Societa Astronomica Italiana
   \newcommand\mnassa{MNASSA}           % Monthly Notes of the Astronomical Society of Southern Africa
   \newcommand\mnras{MNRAS}             % Monthly Notices of the Royal Astronomical Society
   \newcommand\na{New~Astron.}          % New Astronomy
   \newcommand\nar{New~Astron.~Rev.}    % New Astronomy Review
   \newcommand\nat{Nature}              % Nature
   \newcommand\nphysa{Nuclear Phys.~A}  % Nuclear Physics A
   \newcommand\pra{Phys. Rev.~A}        % Physical Review A: General Physics
   \newcommand\prb{Phys. Rev.~B}        % Physical Review B: Solid State
   \newcommand\prc{Phys. Rev.~C}        % Physical Review C
   \newcommand\prd{Phys. Rev.~D}        % Physical Review D
   \newcommand\pre{Phys. Rev.~E}        % Physical Review E
   \newcommand\prl{Phys. Rev.~Lett.}    % Physical Review Letters
   \newcommand\pasa{Publ. Astron. Soc. Australia}  % Publications of the Astronomical Society of Australia
   \newcommand\pasp{PASP}               % Publications of the Astronomical Society of the Pacific
   \newcommand\pasj{PASJ}               % Publications of the Astronomical Society of Japan
   \newcommand\physrep{Phys.~Rep.}      % Physics Reports
   \newcommand\physscr{Phys.~Scr.}      % Physica Scripta
   \newcommand\planss{Planet. Space~Sci.} % Planetary Space Science
   \newcommand\procspie{Proc.~SPIE}     % Proceedings of the Society of Photo-Optical Instrumentation Engineers
   \newcommand\rmxaa{Rev. Mex. Astron. Astrofis.} % Revista Mexicana de Astronomia y Astrofisica
   \newcommand\qjras{QJRAS}             % Quarterly Journal of the RAS
   \newcommand\sci{Science}             % Science
   \newcommand\skytel{Sky \& Telesc.}   % Sky and Telescope
   \newcommand\solphys{Sol.~Phys.}      % Solar Physics
   \newcommand\sovast{Soviet~Ast.}      % Soviet Astronomy (aka Astronomy Reports)
   \newcommand\ssr{Space Sci. Rev.}     % Space Science Reviews
   \newcommand\zap{Z.~Astrophys.}       % Zeitschrift fuer Astrophysik
   
   %  ****************************************
   %  *    COMMANDS FOR USE WITH MNRAS.BST    *
   %  ****************************************
   %
   % The following three macros provide auxiliary support for the BibTeX
   % wranglings in mnras.bst.  They provide support for functionality
   % which it is impossible, or at least unmaintainably arcane, to
   % provide within BibTeX Style Language.
   %
   % These definitions can be loaded as a package or, probably better,
   % should be incorporated into a mnras.cls file.
   %
   % These depend on the presence of a \href{URL}{text} macro, as
   % provided by the hyperref package.  The mnras.bst style depends
   % additionally on the \url{URL} macro, which hyperref also provides.
   %
   % If the hyperref package is not included, then suitable defaults are
   %
   %   \def\href#1#2{#2}
   %   \def\@url#1{#1\endgroup}
   %   \def\url{\begingroup\@urlcharsother \ttfamily \@url}
   %
   % These must appear _after_ this package is loaded, and should appear
   % _instead_ of loading the hyperref package (it'll probably be OK to
   % let the hyperref package redefine these, but that is to tempt fate).
   
   
   % \@urlcharsother
   %
   % 'Other' some characters which may appear in DOIs and URIs.
   %
   % All of the characters here may appear in URIs, except for '^' and '\'.
   %
   % There appear to be almost no restrictions on what characters appear
   % in DOIs (or at least none discovereable in ISO 26324:2012, which
   % says simply that the 'DOI suffix' is "a character string of any
   % length".  A DOI registrant which uses characters outside ASCII plus
   % the following set, is a DOI registrant who should be taken outside
   % and challenged on their taste.
   %
   % The following list is not simply \dospecials, because that includes
   % '{' and '}', which we need.  And if they're in a DOI... well.
   \def\@urlcharsother{%
       \let\do\@makeother 
       \do\\\do\$\do\&\do\#\do\^\do\_\do\%\do\~}
   
   % \doi
   %
   % \doi{10.foo} formats the DOI in the argument, and provides a link to dx.doi.org.
   % \doi[text]{10.foo} formats the DOI 10.foo, but provides 'text' as the link.
   % The DOI can contain {\$&#^_%~} (though there's not necessarily a
   % guarantee that these will still work as URL characters within the PDF)
   \def\doi{\begingroup
       \@urlcharsother
       \@ifnextchar[%
       {\@doi}
       {\@doi[]}}
   \def\@doi[#1]#2{%
       \def\@tempa{#1}%
       \ifx\@tempa\@empty
       \href{http://dx.doi.org/#2}{doi:#2}%
       \else
       \href{http://dx.doi.org/#2}{#1}%
       \fi
       \endgroup
   }
   
   % \eprint
   %
   % \eprint{defaultArchivePrefix}{id} expands to a link to the given ID
   % at a suitable archive.  The 'id' can be either a bare ID (such as
   % yymm.1234) for arXiv, or can include an archive prefix.  If there is
   % no prefix in the 'id', then 'defaultArchivePrefix' supplies a default.
   %
   % Thus
   %   \eprint{}{arXiv:yymm.1234} -> \href{http://arxiv.org/abs/yymm.1234}{arXiv:yymm.1234}
   %   \eprint{}{yymm.1234} -> same as \eprint{}{arXiv:yymm.1234}
   %   \eprint{arXiv}{arXiv:yymm.1234} -> same
   %   \eprint{dblp}{1234} -> \href{http://dblp.uni-trier.de/rec/bibtex/1234.xml}{dblp:1234}
   %   \eprint{dblp}{arXiv:yymm.1234} -> same as \eprint{}{arXiv:yymm.1234}
   %   \eprint{}{wibble:1234} -> wibble:1234 (doesn't match anything)
   %
   % A prefix 'PFX' is 'registered' by defining a macro
   % \@eprint@PFX#1{...} which formats the identifier (that is, \eprint's
   % second argument _minus_ any colon-terminated prefix).
   \def\eprint#1#2{%
       \@eprint#1:#2::\@nil}
   \def\@eprint@arXiv#1{\href{http://arxiv.org/abs/#1}{{\tt arXiv:#1}}}
   \def\@eprint@dblp#1{\href{http://dblp.uni-trier.de/rec/bibtex/#1.xml}{dblp:#1}}
   \def\@eprint#1:#2:#3:#4\@nil{%
       \def\@tempa{#1}%
       \def\@tempb{#2}%
       \def\@tempc{#3}%
       \ifx\@tempc\@empty
       \let\@tempc\@tempb
       \let\@tempb\@tempa
       \fi
       \ifx\@tempb\@empty
       % default to arXiv
       \def\@tempb{arXiv}%
       \fi
       % If \@tempb is a 'recognised' prefix, then call it, otherwise, just
       % print prefix:id and be done with it.  A prefix is 'recognised' if
       % there's a macro \@eprint@<prefix>.
       \@ifundefined{@eprint@\@tempb}
       {\@tempb:\@tempc}
       % or call macro '@eprint@\@tempb' on the argument \@tempc
       {\expandafter\expandafter\csname @eprint@\@tempb\endcsname\expandafter{\@tempc}}%
   }
   
   % \mniiiauthor
   %
   % The following implements the three-author-hack described in mnras.bst.
   %
   % This consumes a command for each such author.  It's surely possible
   % to avoid this (with some constructions involving {\\#1}; see
   % Appendix D cleverness), but that would verge on the unmaintanably
   % arcane, and not really be worth it.
   \def\mniiiauthor#1#2#3{%
       \@ifundefined{mniiiauth@#1}
       {\global\expandafter\let\csname mniiiauth@#1\endcsname\null #2}
       {#3}}
   
   %-----------------------------------------------------------------------------------------------

% If you would like to use biblatex for your reference management, as opposed to the default `natbibpackage` pass the option `custombib` in the document class. Comment out the previous line to make sure you don't load the natbib package. Uncomment the following lines and specify the location of references.bib file

%\usepackage[backend=biber, style=numeric-comp, citestyle=numeric, sorting=nty, natbib=true]{biblatex}
%\addbibresource{References/references} %Location of references.bib only for biblatex, Do not omit the .bib extension from the filename.

\fi

% changes the default name `Bibliography` -> `References'
\renewcommand{\bibname}{References}
% ******************************* Joris' edit **********************************
% (Edit: change font to small and add text)
\renewcommand{\bibfont}{\small}
%\renewcommand{\bibpreamble}{These are the references used\dots}


% ******************************************************************************
% ************************* User Defined Commands ******************************
% ******************************************************************************

% *********** To change the name of Table of Contents / LOF and LOT ************

%\renewcommand{\contentsname}{My Table of Contents}
%\renewcommand{\listfigurename}{My List of Figures}
%\renewcommand{\listtablename}{My List of Tables}


% ********************** TOC depth and numbering depth *************************

\setcounter{secnumdepth}{3}
\setcounter{tocdepth}{3}


% ******************************* Nomenclature *********************************

% To change the name of the Nomenclature section, uncomment the following line

%\renewcommand{\nomname}{Symbols}


% ********************************* Appendix ***********************************

% The default value of both \appendixtocname and \appendixpagename is `Appendices'. These names can all be changed via:

%\renewcommand{\appendixtocname}{List of appendices}
%\renewcommand{\appendixname}{Appndx}

% *********************** Configure Draft Mode **********************************

\ifsetDraft
    % Uncomment to disable figures in `draft'
    %\setkeys{Gin}{draft=true}  % set draft to false to enable figures in `draft'
    
    % These options are active only during the draft mode
    % Default text is "Draft"
    %\SetDraftText{DRAFT}
    
    % Default Watermark location is top. Location (top/bottom)
    \SetDraftWMPosition{bottom}
    
    % Draft Version - default is v1.0
    \SetDraftVersion{v1.0}
    
    % Draft Text grayscale value (should be between 0-black and 1-white)
    % Default value is 0.75
    %\SetDraftGrayScale{0.8}
\fi


% ******************************** Todo Notes **********************************
%% Uncomment the following lines to have todonotes.

\ifsetDraft
	\usepackage[colorinlistoftodos]{todonotes}
	\newcommand{\mynote}[1]{\todo[author=Joris,size=\small,inline,color=green!40]{#1}}
\else
	\newcommand{\mynote}[1]{}
	\newcommand{\listoftodos}{}
\fi

% Example todo: \mynote{Hey! I have a note}

% ******************************** Highlighting Changes **********************************
% Uncomment the following lines to be able to highlight text/modifications.
\ifsetDraft
% ******************************* Joris' edit **********************************
% Don't load color, soul packages (instead always use these, also when not in draft mode)
%  \usepackage{color, soul}
  \newcommand{\hlc}[2][yellow]{{\sethlcolor{#1} \hl{#2}}}
  \newcommand{\hlfix}[2]{\texthl{#1}\todo{#2}}
\else
  \newcommand{\hlc}[2]{}
  \newcommand{\hlfix}[2]{}
\fi

% Example highlight 1: \hlc{Text to be highlighted}
% Example highlight 2: \hlc[green]{Text to be highlighted in green colour}
% Example highlight 3: \hlfix{Original Text}{Fixed Text}

% *****************************************************************************
% ******************* Better enumeration my MB*************
\usepackage{enumitem}
